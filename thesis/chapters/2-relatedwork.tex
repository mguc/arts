\chapter{Related Work}\label{chap:relatedwork}

\textbf{Articles}\newline
\cite{rober2007ray} \textit{Niklas Röber et al. "Ray acoustics using computer graphics technology"} \newline
Most of the ideas for this thesis are based on the geometric acoustic approach of this paper.
For higher frequencies the geometric acoustic approach seems very promising and can be efficiently realized with GPU acceleration.
The room impulse response (RIR) method is especially beneficial for generating test data for TDOA or AOA systems.

\cite{rindel1995computer} \textit{Jens Holger Rindel, "Computer simulation techniques for acoustical design of rooms"} \newline
This paper has been found on the website of ODEON and displays a ray tracing, an image source and a hybrid method for room acoustic simulations.
The presented Image Source Method has been the main driver and approach for ARTS.

\textbf{Implementations}\newline
\cite{odeon} \textit{Odeon A/S, ODEON} \newline
ODEON is a commercial software for simulating and measuring indoor room acoustics of buildings.
Some of the above mentioned articles and their methods have been used to build ODEON.
It features a rich database of materials with absorption coefficients.

\cite{wayverb} \textit{Reuben Thomas, Wayverb} \newline
Wayverb is an open source software for room acoustics, similar to ODEON.
Some of the ideas have been used in this thesis as well, especially the geometric image-source method.
It is currently only supported on OSX and needs porting for other platforms.
Unfortunately it has been discovered pretty late in the project process.
It would have been a very good starting point as it seems to be very promising and worth looking into for indoor localization and mapping problems.
