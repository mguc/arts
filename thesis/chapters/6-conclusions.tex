\chapter{Conclusion and Future Work}\label{chap:conclusion}
The Acoustic Ray Tracing approach seems very promising for indoor localization and mapping applications.
With ARTS a very useful and scalable tool for indoor environments has been created.
The modular architecture provides flexibility and lots of room for exploration.
The tool can be used on any platform by containerizing it with tools like Docker\cite{docker}.
This should be considered especially for future releases including GPU acceleration.
The use of industry standard 3D model formats allows any user to rapidly design and test arbitrary environments with tools of their choice.
Once creating these 3D models can be automated or accelerated, the need for a tool like ARTS for indoor localization or mapping applications becomes indispensable.

There are certain functionalities, improvements and features missing which might be necessary and great to add:

The run-time of the Ray Tracing Module is still slow and it has a lot of potential to be optimized.
The same can be said for the Path Sampler.
Since these parts of the algorithm can be parallelized, we are almost guaranteed to see an improvement in performance by using a GPU.
This would most likely require moving away from CGAL\cite{cgal} as it currently doesn't support accessing CGAL objects from multiple threads.
The best solution would probably be to use data structures and API calls from libraries like OpenCL or Vulkan \cite{vulkan}, which support multiple GPUs, even on embedded platforms.
The docker functionality for such solutions is probably limited or currently not possible.
CUDA\cite{cuda} seems to work for Linux based systems but limited to NVIDIA GPUs.

Another area worth exploring is possibly leveraging the compute power of NVIDIA's new RTX GPUs\cite{nvidia}.
If they can do light ray tracing in real-time it will probably work for acoustics as well.

The proposed model does not include effects like scattering, diffraction and the influence of signal frequency.
Continued work that takes these effects into consideration would likely yield a more realistic model, especially in regards to noise generation.
Given the modular architecture, it should be possible to gradually add these effects to the model.
